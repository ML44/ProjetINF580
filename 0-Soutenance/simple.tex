
% Simple beamer file.  Run Latex on it in the usual way.


\documentclass[unknownkeysallowed]{beamer}
\usepackage[utf8]{inputenc}       	%% These two packages are required
\usepackage[french]{babel}	%% to type in French with accents.
\usepackage{epstopdf}			%% Package to convert .eps (encapsulated postscript files to pdf)
\usepackage{tikz}			%% Image manipulation to change opacity, position etc. for watermarks!
\usepackage{amsmath, amsthm, amssymb}	%% Normal math-y things
\usepackage{nicefrac}			%% Really sexy package to make sexy fractions.
\usepackage{listings} 		%% To write code
\usepackage{color}
\usepackage{multicol}
\setlength{\columnseprule}{0.5pt}
\usepackage[T1]{fontenc}

%% [[ DEFINING `POLYTECHNIQUE' COLOURS
\definecolor{Blue}{RGB}{0,59,92}
\definecolor{Grey}{RGB}{74,73,72}
\definecolor{Red}{RGB}{247,38,22}
\definecolor{DarkRed}{RGB}{197,26,27}
\definecolor{Yellow}{RGB}{223,176,0}
\definecolor{White}{RGB}{255,255,255}
%% END POLYTECHNIQUE COLORS]]



%%[[ SETTING COLOURS FOR DIFFERENT PROPERTIES 
%% Check out the beamer cheatsheet (Google is your friend) to find the options like 'title' etc.
\setbeamercolor{title}{fg=Blue}					%% Colour of title text
\setbeamercolor{frametitle}{fg=Blue}				%% Color of frame titles
\setbeamercolor{structure}{fg=Blue}
\setbeamercolor{itemize/enumerate body}{fg=Grey}		%% Colour of words in an itemised list
\setbeamercolor{itemize/enumerate subbody}{fg=black}		%% Colour of words in an itemised sublist
\setbeamercolor{itemize subitem}{fg=Grey}			%% Colour of bullet points in the sublist
\setbeamercolor{normal text}{fg=DarkRed}			%% Colour of normal text
\setbeamercolor{section in toc}{fg=Grey}			%% Colour of sections in the Table of Contents
\setbeamercolor{institute}{fg=Blue}				%% Colour of institute in Table of Contents
\setbeamercolor{footlinecolor}{fg=White,bg=Blue}		%% Colour of footline bar
%% END SETTING COLOURS FOR PROPERTIES]]




%% [[CODE FOR HORIZONTAL LINE UNDER TITLE
%% The header in this template is simply a horizontal line. The addition of a subtitle will 
%% make the presentation ugly (as the line will pass through it) and thus you would need to 
%% change the y coordinate of the line. This is done at the line:
%% \pgfline{\pgfpoint{x-start}{y-start}}{\pgfpoint{x-end}{y-end}}
%% where (x-start,y-start) and (x-end,y-end) are the beginning and end points of the line.

\newcommand{\LogoLine}{%				%% Define a bunch of characteristics we call \Logoline
\raisebox{-12mm}[0pt][0pt]{%
\begin{pgfpicture}{0mm}{0mm}{0mm}{0mm}
\pgfsetlinewidth{0.28mm}
\color{Grey}
\pgfline{\pgfpoint{-6.5mm}{1mm}}{\pgfpoint{10cm}{1mm}}
\end{pgfpicture}}}

\setbeamertemplate{headline}[text line]{\LogoLine } 	%% Use \Logoline to make a header.
%% END CODE FOR HORIZONTAL LINE UNDER TITLE]]


%% [[CODE FOR FOOTINE				%% We could do this as before with the header, but we combine the two steps into one directly.
\makeatletter					%% This is required to make the '@' symbol a character, we'll revert this later.
\setbeamertemplate{footline}
{
  \leavevmode%
  \hbox{%					%% Create a horizontal box for the total page length.
 \begin{beamercolorbox}[wd=.5\paperwidth,ht=2.25ex,dp=1ex,center]{footlinecolor}%% Within this box, create a beamercolorbox with [width,height,depth]
    \usebeamerfont{title in head/foot}\insertsection %% Insert the name of the section
  \end{beamercolorbox}% 			%% End the box. (Note, the '%' after '\end{beamercolorbox} is necessary to create a carriage return.
  
  \begin{beamercolorbox}[wd=.5\paperwidth,ht=2.25ex,dp=1ex]{footlinecolor}%    %% Create a second box, etc.
    \usebeamerfont{title in head/foot}\inserttitle \hfill \insertframenumber \,\,\,\,%
  \end{beamercolorbox}}%
  \vskip0pt%
}
\makeatother					%% We make the '@' symbol back to a \TeX\ command
%% END CODE FOR FOOTLINE]]


\setbeamertemplate{navigation symbols}{} 	%% Remove navigation symbols. (I hate them, I hate them, I hate them...)

\setbeamertemplate{itemize items}[circle]	%% Use circle bullet points (triangles are default, you can also use 'square' and 'ball'.) 



\title{Mathematical Programming : \\Writing poetry}	%% Pretty self-explanatory, no?
\author{Matthieu Lequesne \& Quentin Lisack}						%%
\institute{École polytechnique}					%%
\date[]{24 février 2016}


\newcommand{\high}[1]{\textcolor{DarkRed}{#1}}			%% I sometimes need to highlight things. Usually I make 'em bold, sometimes not.


\AtBeginSection[] {				
% Comment out this section if  you don't want the table of contents before EVERY section
{						
% If so, make sure you copy *everything* after the \AtBeginSection and put it where you want the TOC.
%% Large polytechnique logo on the right hand side of the page ONLY for the Table of Contents. (That's why it's in the {}s )
\setbeamertemplate{background canvas}{
	\tikz[remember picture,overlay]
	{
		\node[opacity=0.2] at (current page.east) 	
		{\includegraphics[width=8cm,keepaspectration]{./img/blazon}} ;
	}
}
\setbeamertemplate{footline}{}  %% Removes footline for the table of contents page, 'cause it's UGLY
\begin{frame}
 \frametitle{Table des matières}
 \tableofcontents[currentsection]
 \addtocounter{framenumber}{-1}% You don't want them to affect the slide number, do you?
\end{frame}
}
}



%%%%% For JAVA

\definecolor{javared}{rgb}{0.6,0,0} % for strings
\definecolor{javagreen}{rgb}{0.25,0.5,0.35} % comments
\definecolor{javapurple}{rgb}{0.5,0,0.35} % keywords
\definecolor{javadocblue}{rgb}{0.25,0.35,0.75} % javadoc
 
\lstset{language=Java,
basicstyle=\ttfamily\scriptsize,
keywordstyle=\color{javapurple}\bfseries,
stringstyle=\color{javared},
commentstyle=\color{javagreen},
morecomment=[s][\color{javadocblue}]{/**}{*/},
numbers=left,
numberstyle=\tiny\color{black},
stepnumber=1,
numbersep=10pt,
tabsize=4,
showspaces=false,
showstringspaces=false
breaklines=true
}


\begin{document}
% Title slide
{
%% The next two lines define the logo on the first page using the tikz package to add an image to the template of the 'background canvas'.
%% The \node[] function tells it *where* to place the center of the image. Thus, \node[] at (current page.south east) places the image center 
%% at the south east corner. We want it to be slightly off-centered, and so we use the \node[] function with a 'shift' argument, where 
%% \node[shift={(x,y)}] shifts the node by a distance -x, -y away from the (in this case) south east point.  The option opacity controls, well.. ;P
%%\setbeamertemplate{background canvas}{
%%\tikz[remember picture,overlay]\node[shift={(-1.25cm,1.25cm)},opacity=0.8] at (current page.south east) {\includegraphics[width=8cm, keepaspectration]{./img/escher1.png}};
% \tikz[remember picture,overlay]\node[shift={(0cm,-1.5cm)},opacity=1] at (current page.north) {\includegraphics[width=0.75cm,keepaspectration]{./img/EPred}};
%}
\frame[plain]{\titlepage}
}



% First non-title slide
%% Use Polytechnique coat of arms at the top right corner for all slides after this, between the following { }s.

\setbeamertemplate{background canvas}{
	\tikz[remember picture,overlay]
	\node[shift={(-1cm,-1.25cm)},opacity=0.5] at (current page.north east) 			
	{\includegraphics[width=1.25cm, keepaspectration]{./img/blazon}}
	;
}



\section{Our goal}

\begin{frame}{Our goal}


\begin{quote}
\og Don’t use the phone. People are never ready to answer it. Use poetry. \fg
\flushright \textsc{Jack Kerouack}
\end{quote}

\begin{itemize}
\item Writing two verses.
\item Each verse is 12 syllables long.
\item Last words rhyme together.
\item Sentences make sense.
\end{itemize}
\end{frame}



\section{Implementation}


\begin{frame}[fragile]{First try}
\begin{itemize}
\item Index words
\item Rhymes
\begin{itemize}
\item Each word belongs to a rhyme class
\item The last words must belong to the same class
\end{itemize}
\item Syllables
\begin{itemize}
\item For each word have the number of syllables
\item The sum of syllables in each verse has to be 12
\end{itemize}
\item Make sense
\begin{itemize}
\item Read a long text
\item Compute how frequently one word follows another
\item Maximize the sum of frequencies 
\end{itemize}
\end{itemize}
\end{frame}

\begin{frame}[fragile]{First try}
\begin{itemize}
\item .dat files
\item Table of words
\item The empty word
\item Linearization
\end{itemize}

Problems: 

\begin{itemize}
\item Deterministic
\item Time of the universe
\end{itemize}
\end{frame}

\begin{frame}[fragile]{Second try}
\begin{itemize}
\item Choose the rhyme randomly
\medskip
\item Split and overlap : two tables of 7 words
\end{itemize}
\end{frame}

\begin{frame}[fragile]{Another approach}
\begin{itemize}
\item Syllable by syllable
\medskip

\begin{itemize}
\item Drop the 12-syllables constraint
\medskip

\item Need to have real words ? 
\medskip

\begin{itemize}
\item Too strong
\medskip

\item Look like real words
\end{itemize}
\end{itemize}
\end{itemize}
\end{frame}


\begin{frame}[fragile]{Sequencial approach}
\begin{itemize}
\item Construct the sentence word by word from the end
\medskip

\item Knapsack problem
\medskip

\item Works pretty well
\end{itemize}
\end{frame}


\begin{frame}[fragile]{Combining strategies}
\begin{itemize}
\item Construct the sentence 3 words by 3 words from the end
\medskip

\item Best achievement so far, but still too long
\end{itemize}
\end{frame}

\section{Conclusion}

\begin{frame}[fragile]{Conclusion}
\begin{quote}
\small
\og LA PROSE ET DE L ON NOMME ETIQUETTE \\
 PROSE ET J AI VU PAR TOUS LES CIEUX ALOUETTE \fg
\end{quote}

\begin{itemize}
\item Have better definitions of:
\begin{itemize}
\item The number of syllables (too rough)
\item Liaisons ...
\item The rhyme classes (too strong)
\end{itemize}
\item Have more powerful solvers
\item Read longer texts
\end{itemize}
\end{frame}



\end{document}
